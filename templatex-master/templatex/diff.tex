%%
%DIF LATEXDIFF DIFFERENCE FILE
%DIF DEL poster.tex    Thu Oct 10 15:04:10 2019
%DIF ADD poster1.tex   Tue Nov 19 16:55:38 2019
%% This is file `tikzposter-template.tex',
%% generated with the docstrip utility.
%%
%% The original source files were:
%%
%% tikzposter.dtx  (with options: `tikzposter-template.tex')
%%
%% This is a generated file.
%%
%% Copyright (C) 2014 by Pascal Richter, Elena Botoeva, Richard Barnard, and Dirk Surmann
%%
%% This file may be distributed and/or modified under the
%% conditions of the LaTeX Project Public License, either
%% version 2.0 of this license or (at your option) any later
%% version. The latest version of this license is in:
%%
%% http://www.latex-project.org/lppl.txt
%%
%% and version 2.0 or later is part of all distributions of
%% LaTeX version 2013/12/01 or later.
%%


\documentclass{tikzposter} %Options for format can be included here

\usepackage{todonotes}

\usepackage[tikz]{bclogo}
\usepackage{lipsum}
\usepackage{amsmath}

\usepackage{booktabs}
\usepackage{longtable}
\usepackage[absolute]{textpos}
\usepackage[it]{subfigure}
\usepackage{graphicx}
\usepackage{cmbright}
%\usepackage[default]{cantarell}
%\usepackage{avant}
%\usepackage[math]{iwona}
\usepackage[math]{kurier}
\usepackage[T1]{fontenc}


%% add your packages here
\usepackage{hyperref}
% for random text
\usepackage{lipsum}
\usepackage[english]{babel}
\usepackage[pangram]{blindtext}

\colorlet{backgroundcolor}{blue!10}

 % Title, Author, Institute
\title{Group Outlying Aspects Mining}
\author{Shaoni Wang$^1$, Gang Li$^2$}
\institute{$^1$ Xi'an Shiyou University, China \\
	$^2$ Deakin University, Australia
}
%\titlegraphic{logos/tulip-logo.eps}

%Choose Layout
\usetheme{Wave}

%\definebackgroundstyle{samplebackgroundstyle}{
%\draw[inner sep=0pt, line width=0pt, color=red, fill=backgroundcolor!30!black]
%(bottomleft) rectangle (topright);
%}
%
%\colorlet{backgroundcolor}{blue!10}
%DIF PREAMBLE EXTENSION ADDED BY LATEXDIFF
%DIF UNDERLINE PREAMBLE %DIF PREAMBLE
\RequirePackage[normalem]{ulem} %DIF PREAMBLE
\RequirePackage{color}\definecolor{RED}{rgb}{1,0,0}\definecolor{BLUE}{rgb}{0,0,1} %DIF PREAMBLE
\providecommand{\DIFaddtex}[1]{{\protect\color{blue}\uwave{#1}}} %DIF PREAMBLE
\providecommand{\DIFdeltex}[1]{{\protect\color{red}\sout{#1}}}                      %DIF PREAMBLE
%DIF SAFE PREAMBLE %DIF PREAMBLE
\providecommand{\DIFaddbegin}{} %DIF PREAMBLE
\providecommand{\DIFaddend}{} %DIF PREAMBLE
\providecommand{\DIFdelbegin}{} %DIF PREAMBLE
\providecommand{\DIFdelend}{} %DIF PREAMBLE
%DIF FLOATSAFE PREAMBLE %DIF PREAMBLE
\providecommand{\DIFaddFL}[1]{\DIFadd{#1}} %DIF PREAMBLE
\providecommand{\DIFdelFL}[1]{\DIFdel{#1}} %DIF PREAMBLE
\providecommand{\DIFaddbeginFL}{} %DIF PREAMBLE
\providecommand{\DIFaddendFL}{} %DIF PREAMBLE
\providecommand{\DIFdelbeginFL}{} %DIF PREAMBLE
\providecommand{\DIFdelendFL}{} %DIF PREAMBLE
%DIF HYPERREF PREAMBLE %DIF PREAMBLE
\providecommand{\DIFadd}[1]{\texorpdfstring{\DIFaddtex{#1}}{#1}} %DIF PREAMBLE
\providecommand{\DIFdel}[1]{\texorpdfstring{\DIFdeltex{#1}}{}} %DIF PREAMBLE
\newcommand{\DIFscaledelfig}{0.5}
%DIF HIGHLIGHTGRAPHICS PREAMBLE %DIF PREAMBLE
\RequirePackage{settobox} %DIF PREAMBLE
\RequirePackage{letltxmacro} %DIF PREAMBLE
\newsavebox{\DIFdelgraphicsbox} %DIF PREAMBLE
\newlength{\DIFdelgraphicswidth} %DIF PREAMBLE
\newlength{\DIFdelgraphicsheight} %DIF PREAMBLE
% store original definition of \includegraphics %DIF PREAMBLE
\LetLtxMacro{\DIFOincludegraphics}{\includegraphics} %DIF PREAMBLE
\newcommand{\DIFaddincludegraphics}[2][]{{\color{blue}\fbox{\DIFOincludegraphics[#1]{#2}}}} %DIF PREAMBLE
\newcommand{\DIFdelincludegraphics}[2][]{% %DIF PREAMBLE
\sbox{\DIFdelgraphicsbox}{\DIFOincludegraphics[#1]{#2}}% %DIF PREAMBLE
\settoboxwidth{\DIFdelgraphicswidth}{\DIFdelgraphicsbox} %DIF PREAMBLE
\settoboxtotalheight{\DIFdelgraphicsheight}{\DIFdelgraphicsbox} %DIF PREAMBLE
\scalebox{\DIFscaledelfig}{% %DIF PREAMBLE
\parbox[b]{\DIFdelgraphicswidth}{\usebox{\DIFdelgraphicsbox}\\[-\baselineskip] \rule{\DIFdelgraphicswidth}{0em}}\llap{\resizebox{\DIFdelgraphicswidth}{\DIFdelgraphicsheight}{% %DIF PREAMBLE
\setlength{\unitlength}{\DIFdelgraphicswidth}% %DIF PREAMBLE
\begin{picture}(1,1)% %DIF PREAMBLE
\thicklines\linethickness{2pt} %DIF PREAMBLE
{\color[rgb]{1,0,0}\put(0,0){\framebox(1,1){}}}% %DIF PREAMBLE
{\color[rgb]{1,0,0}\put(0,0){\line( 1,1){1}}}% %DIF PREAMBLE
{\color[rgb]{1,0,0}\put(0,1){\line(1,-1){1}}}% %DIF PREAMBLE
\end{picture}% %DIF PREAMBLE
}\hspace*{3pt}}} %DIF PREAMBLE
} %DIF PREAMBLE
\LetLtxMacro{\DIFOaddbegin}{\DIFaddbegin} %DIF PREAMBLE
\LetLtxMacro{\DIFOaddend}{\DIFaddend} %DIF PREAMBLE
\LetLtxMacro{\DIFOdelbegin}{\DIFdelbegin} %DIF PREAMBLE
\LetLtxMacro{\DIFOdelend}{\DIFdelend} %DIF PREAMBLE
\DeclareRobustCommand{\DIFaddbegin}{\DIFOaddbegin \let\includegraphics\DIFaddincludegraphics} %DIF PREAMBLE
\DeclareRobustCommand{\DIFaddend}{\DIFOaddend \let\includegraphics\DIFOincludegraphics} %DIF PREAMBLE
\DeclareRobustCommand{\DIFdelbegin}{\DIFOdelbegin \let\includegraphics\DIFdelincludegraphics} %DIF PREAMBLE
\DeclareRobustCommand{\DIFdelend}{\DIFOaddend \let\includegraphics\DIFOincludegraphics} %DIF PREAMBLE
\LetLtxMacro{\DIFOaddbeginFL}{\DIFaddbeginFL} %DIF PREAMBLE
\LetLtxMacro{\DIFOaddendFL}{\DIFaddendFL} %DIF PREAMBLE
\LetLtxMacro{\DIFOdelbeginFL}{\DIFdelbeginFL} %DIF PREAMBLE
\LetLtxMacro{\DIFOdelendFL}{\DIFdelendFL} %DIF PREAMBLE
\DeclareRobustCommand{\DIFaddbeginFL}{\DIFOaddbeginFL \let\includegraphics\DIFaddincludegraphics} %DIF PREAMBLE
\DeclareRobustCommand{\DIFaddendFL}{\DIFOaddendFL \let\includegraphics\DIFOincludegraphics} %DIF PREAMBLE
\DeclareRobustCommand{\DIFdelbeginFL}{\DIFOdelbeginFL \let\includegraphics\DIFdelincludegraphics} %DIF PREAMBLE
\DeclareRobustCommand{\DIFdelendFL}{\DIFOaddendFL \let\includegraphics\DIFOincludegraphics} %DIF PREAMBLE
%DIF END PREAMBLE EXTENSION ADDED BY LATEXDIFF

\begin{document}


\colorlet{blocktitlebgcolor}{blue!23}

 % Title block with title, author, logo, etc.
\maketitle

\begin{columns}
 % FIRST column
\column{0.5}% Width set relative to text width

%%%%%%%%%% -------------------------------------------------------------------- %%%%%%%%%%
 %\block{Main Objectives}{
%  	      	\begin{enumerate}
%  	      	\item Formalise research problem by extending \emph{outlying aspects mining}
%  	      	\item Proposed \emph{GOAM} algorithm is to solve research problem
%  	      	\item Utilise pruning strategies to reduce time complexity
%  	      	\end{enumerate}
%%  	      \end{minipage}
%}
%%%%%%%%%% -------------------------------------------------------------------- %%%%%%%%%%


%%%%%%%%%% -------------------------------------------------------------------- %%%%%%%%%%
\DIFdelbegin %DIFDELCMD < \block{Introduction}{
%DIFDELCMD <     Many real world applications call for one important function
%DIFDELCMD <     of identifying the set of features
%DIFDELCMD <     on which the interested object is most distinguished from others.
%DIFDELCMD <     Usually,
%DIFDELCMD <     this object is termed as the query object,
%DIFDELCMD <     and the set of features are referred to as the \emph{subspaces} or \emph{aspects}.
%DIFDELCMD <     Accordingly,
%DIFDELCMD <     this research problem is referred to as
%DIFDELCMD <     \emph{outlying aspects mining},
%DIFDELCMD <     which is different from \emph{outlier detection}.
%DIFDELCMD <   	

%DIFDELCMD <   	\begin{description}
\begin{description}%DIFAUXCMD
%DIFDELCMD <   	\item[Outlying Aspects Mining] aims to identify a subspace
%DIFDELCMD <     which makes the query object most outlying,
%DIFDELCMD <     rather than verifying whether it is an outlier or not.
%DIFDELCMD <     The task of \emph{Outlying Aspects Mining}
%DIFDELCMD <     is to explain which aspects make the query object most different.
%DIFDELCMD <   	

%DIFDELCMD <   	\item[Outlier Detection] aims to identify all possible outliers in the dataset,
%DIFDELCMD <     without explaining why or how they are different.
%DIFDELCMD <     Hence,
%DIFDELCMD <     the outlying aspects mining is also referred to
%DIFDELCMD <     \emph{outlier interpretation}
%DIFDELCMD <     or \emph{object explanation}.

\end{description}%DIFAUXCMD
%DIFDELCMD <   	\end{description}
%DIFDELCMD < 

%DIFDELCMD <   	In this paper,
%DIFDELCMD <     we extend the task of \emph{outlying aspects mining} to the \emph{group} level,
%DIFDELCMD <     formalize the research problem of \emph{group outlying aspects mining},
%DIFDELCMD <     and propose a novel algorithm named GOAM to solve the
%DIFDELCMD <     \emph{group outlying aspects mining} problem.
%DIFDELCMD < }
%DIFDELCMD < %%%
\DIFdelend \DIFaddbegin \block{Introduction}{
    Many real world applications call for many important functions
    of identifying the set of features
    on which the interested object is most distinguished from others.
    Usually,
    this object is termed as the query object,
    and the set of features are referred to as the \emph{subspaces} or \emph{aspects}.
    Accordingly,
    this research problem is referred to as
    \emph{outlying aspects mining},
    which is different from \emph{outlier detection}.

  	\begin{description}
  	\item[Outlying Aspects Mining] aims to identify a subspace
    which makes the query object most outlying,
    rather than verifying whether it is an outlier or not.
    The task of \emph{Outlying Aspects Mining}
    is to explain which aspects make the query object most different.

  	\item[Outlier Detection] aims to identify all possible outliers in the dataset,
    without explaining why or how they are different.
    Hence,
    the outlying aspects mining is also referred to
    \emph{outlier interpretation}
    or \emph{object explanation}.
  	\end{description}

  	In this paper,
    we extend the task of \emph{outlying aspects mining} to the \emph{group} level,
    formalize the research problem of \emph{group outlying aspects mining},
    and propose a novel algorithm named GOAM to solve the
    \emph{group outlying aspects mining} problem.
}
\DIFaddend %%%%%%%%%% -------------------------------------------------------------------- %%%%%%%%%%


%%%%%%%%%% -------------------------------------------------------------------- %%%%%%%%%%
\block{Group Outlying Aspects Mining}{
\begin{itemize}
    \item
    %\emph{Group Outlying Aspects Mining}
    It aims to \emph{identify a subset of aspects (or subspace)
    which makes the query group, rather than the single object,
    obviously different}.
    What we are interested in the task of \emph{group outlying aspects mining}
    is to explain which aspects make the query group distinctive
    different from the other groups.

    \item
    \emph{Group Outlying Aspects Mining},
    \emph{Outlying Aspects Mining} and
    \emph{Outlier Detection} are different with each other.
\end{itemize}

\begin{center}
    \begin{minipage}{0.3\linewidth}
    \centering
    \begin{tikzfigure}
    \missingfigure[figcolor=white]{Testing figcolor}
    {\small{Group Outlying Aspects Mining}}
    \end{tikzfigure}%
    \end{minipage}
    \hfill
    \begin{minipage}{0.3\linewidth}
    \centering
    \begin{tikzfigure}
    \missingfigure[figcolor=white]{Testing figcolor}
    {\small{Outlying Aspects Mining}}
    \end{tikzfigure}%
    \end{minipage}
    \hfill
    \begin{minipage}{0.3\linewidth}
    \centering
    \begin{tikzfigure}
    \missingfigure[figcolor=white]{Testing figcolor}
    {\small{Outlier Detection}}
    \end{tikzfigure}%
    \end{minipage}
\end{center}
}
%%%%%%%%%% -------------------------------------------------------------------- %%%%%%%%%%


%%%%%%%%%% -------------------------------------------------------------------- %%%%%%%%%%

%\note{Note with default behavior}

%\note[targetoffsetx=12cm, targetoffsety=-1cm, angle=20, rotate=25]
%{Note \\ offset and rotated}

 % First column - second block


%%%%%%%%%% -------------------------------------------------------------------- %%%%%%%%%%
\block{GOAM Algorithm}{
  	We propose the \emph{GOAM} algorithm to solve the research problem of
    \emph{Group Outlying Aspects Mining}.
  	The \emph{GOAM} algorithm includes three major steps.
%    1) Group Feature Extraction,
%    2) Outlying Degree Scoring, and
%    3) Outlying Aspects Identification.

\begin{tikzfigure}%[Overall architecture of \emph{GOAM} algorithm]
%  \includegraphics[width=0.8\linewidth]{figures//framework.pdf}
    \missingfigure[figcolor=white]{Testing figcolor}
\end{tikzfigure}

\begin{description}
  	\item[Group Feature Extraction]
  	Let $f_1$, $f_2$, $f_3$ represent three features of $G_q$.
    We count the frequency of each value for one feature.
    Then use the histogram to represent each feature.
    Similarly,
    we can extract other features for each group.

%    \item
%    The histogram of $G_q$ on three features are as follows.
\end{description}

\begin{center}
    \begin{minipage}{0.3\linewidth}
    \centering
    \begin{tikzfigure}
    \missingfigure[figcolor=white]{Testing figcolor}
    {\small{Histogram of $G_q$ on $f_1$}}
    \end{tikzfigure}%
    \end{minipage}
    \hfill
    \begin{minipage}{0.3\linewidth}
    \centering
    \begin{tikzfigure}
    \missingfigure[figcolor=white]{Testing figcolor}
    {\small{Histogram of $G_q$ on $f_2$}}
    \end{tikzfigure}%
    \end{minipage}
    \hfill
    \begin{minipage}{0.3\linewidth}
    \centering
    \begin{tikzfigure}
    \missingfigure[figcolor=white]{Testing figcolor}
    {\small{Histogram of $G_q$ on $f_3$}}
    \end{tikzfigure}%
    \end{minipage}
\end{center}
\begin{description}
\item[Outlying Degree Scoring]
    In this step,
    we first calculate the \emph{earth mover distance} (EMD) of one feature among different groups.
    The earth mover distance reflects the minimum mean distance
    between groups on one feature.
    So,
    we utilize the EMD to measure the difference between groups of each feature.
\end{description}
}
%%%%%%%%%% -------------------------------------------------------------------- %%%%%%%%%%


% SECOND column
\column{0.5}
 %Second column with first block's top edge aligned with with previous column's top.

%%%%%%%%%% -------------------------------------------------------------------- %%%%%%%%%%
\block{GOAM Algorithm}{
\begin{description}
    \item
    Second,
    based on the \emph{earth move distance},
    we calculate the outlying degree.
\end{description}

\begin{tikzfigure}%[Overall architecture of \emph{GOAM} algorithm]
    \missingfigure[figcolor=white]{Testing figcolor}
\end{tikzfigure}
  where $G_q$ is the query group,
  $n$ is the number of compare groups,
  and $h_{k_s}$ is the histogram representation of $G_k$ in the subspace $s$.

\begin{description}
  	\item[Outlying Aspects Identification]
    In this step,
    based on the value of outlying degree
    we will identify the group outlying aspects.
    If a feature's outlying degree is greater than a threshold,
    the more likely the feature is group outlying aspect.
    When the dimensionality of features is high,
    we adopt a stage-wise candidate subspace construction strategy to
    alleviate the exponential explosion.
\end{description}
}
%%%%%%%%%% -------------------------------------------------------------------- %%%%%%%%%%
% Second column - first block


%%%%%%%%%% -------------------------------------------------------------------- %%%%%%%%%%
\block[titleleft]{Experiment}
{
\begin{description}
  	\item[Synthetic Dataset] contains $10$ groups and $8$ features.
    Each group consists of $10$ members,
    and each member has $8$ features.
\end{description}
\vspace{.5cm}
\begin{tabular}{ c | c | c | c }
    \toprule
    Method     &  Truth Outlying Aspects    & Identified Aspects & Accuracy      \\
    \midrule
    GOAM       &  $\{F_1\}$, $\{F_2F_4\}$   &  $\{F_1\}$, $\{F_2F_4\}$    & 100\%    \\

     Arithmetic Mean based OAM &  $\{F_1\}$, $\{F_2F_4\}$   &  $\{F_4\}$, $\{F_2\}$    &  0\% \\

     Median based OAM &  $\{F_1\}$, $\{F_2F_4\}$   &  $\{F_2\}$, $\{F_4\}$    &           0\% \\
     \bottomrule
\end{tabular}
\vspace{.2cm}
\begin{description}
    \item
    It can be observed that the GOAM method can identify the trivial outlying features
    and non-trivial outlying subspaces correctly and is obvious from the table
    that the accuracy of GOAM is the best, which is ($100\%$).
\end{description}

\begin{description}
\item[NBA Dataset] was collected from Yahoo Sports
website (\url{http://sports.yahoo.com.cn/nba}).
The data include all teams from the six divisions,
and each player in the team has $12$ features.
\end{description}
\vspace{.5cm}
\begin{tabular}{ c | c | c }
    \toprule
    Teams                   & Trivial Outlying Aspects  & NonTrivial Outlying Aspects    \\
    \toprule
    Cleveland Cavaliers     & \{3FA\}                   & \{FGA, FT\%\}, \{FGA, FG\%\} \\
    Orlando Magic           & \{Stl\}                   & None                         \\
    Milwaukee Bucks         & \{To\}, \{FTA\}           & \{FGA, FTA\}, \{3FA, FTA\}     \\
%    Golden State Warriors   & \{FG\%\}                  & \{FT\%, Blk\}, \{FGA, 3PT\%, FTA\}\\
%    Utah Jazz               & \{Blk\}                   & \{3FA, 3PT\%\}                    \\
    New Orleans Pelicans    & \{FT\%\}, \{FTA\}         & \{FTA, Stl\}, \{FTA, To\}          \\
    \bottomrule
\end{tabular}

\begin{minipage}{0.5\linewidth}
    \centering
    \begin{tikzfigure}
    \missingfigure[figcolor=white]{Testing figcolor}

    {\small{New Orleans Pelicans on FT\%}}
    \end{tikzfigure}%
\end{minipage}
\hfill
\begin{minipage}{0.5\linewidth}
    \centering
    \begin{tikzfigure}
    \missingfigure[figcolor=white]{Testing figcolor}

    {\small{New Orleans Pelicans on FTA}}
    \end{tikzfigure}%
\end{minipage}
\vspace{.2cm}
\begin{description}
\item
\texttt{New Orleans Pelicans} has more players with
lower \{free throw percentage\}, \{free throws attempted\}.
\end{description}
}
%%%%%%%%%% -------------------------------------------------------------------- %%%%%%%%%%


% Second column - second block
%%%%%%%%%% -------------------------------------------------------------------- %%%%%%%%%%
\block[titlewidthscale=1, bodywidthscale=1]
{Conclusion}
{
\begin{description}
  \item[Problem Definition]
  Formalize the problem of Group Outlying Aspects Mining by extending outlying aspects mining.

  \item[GOAM algorithm]
  Propose GOAM algorithm to solve the \emph{Group}\\
  \emph{Outlying Aspects Mining} problem.

  \item[Strategies]
  Utilize the pruning strategies to \\ reduce time complexity.
\end{description}
}
%%%%%%%%%% -------------------------------------------------------------------- %%%%%%%%%%


% Bottomblock
%%%%%%%%%% -------------------------------------------------------------------- %%%%%%%%%%
\colorlet{notebgcolor}{blue!20}
\colorlet{notefrcolor}{blue!20}
\note[targetoffsetx=8cm, targetoffsety=-4cm, angle=30, rotate=15,
radius=2cm, width=.26\textwidth]{
Acknowledgement
\begin{itemize}
    \item
    International Cooperation Project (Y7Z0511101)
    of IIE,
    Chinese Academy of Sciences
 \end{itemize}
}

%\note[targetoffsetx=8cm, targetoffsety=-10cm,rotate=0,angle=180,radius=8cm,width=.46\textwidth,innersep=.1cm]{
%Acknowledgement
%}

%\block[titlewidthscale=0.9, bodywidthscale=0.9]
%{Acknowledgement}{
%}
%%%%%%%%%% -------------------------------------------------------------------- %%%%%%%%%%

\end{columns}


%%%%%%%%%% -------------------------------------------------------------------- %%%%%%%%%%
%[titleleft, titleoffsetx=2em, titleoffsety=1em, bodyoffsetx=2em,%
%roundedcorners=10, linewidth=0mm, titlewidthscale=0.7,%
%bodywidthscale=0.9, titlecenter]

%\colorlet{noteframecolor}{blue!20}
\colorlet{notebgcolor}{blue!20}
\colorlet{notefrcolor}{blue!20}
\note[targetoffsetx=-13cm, targetoffsety=-12cm,rotate=0,angle=180,radius=8cm,width=.96\textwidth,innersep=.4cm]
{
\begin{minipage}{0.3\linewidth}
\centering
\includegraphics[width=24cm]{logos/tulip-wordmark.eps}
\end{minipage}
\begin{minipage}{0.7\linewidth}
{ \centering
 The $11^{th}$ International Conference on Knowledge Science,
  Engineering and Management (KSEM 2018),
  17-19/08/2018, Changchun, China
}
\end{minipage}
}
%%%%%%%%%% -------------------------------------------------------------------- %%%%%%%%%%


\end{document}

%\endinput
%%
%% End of file `tikzposter-template.tex'.
